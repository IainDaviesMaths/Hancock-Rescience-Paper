% DO NOT EDIT - automatically generated from metadata.yaml

\def \codeURL{https://github.com/IainDaviesMaths/Reproduction-Hancock.git}
\def \codeDOI{}
\def \codeSWH{}
\def \dataURL{}
\def \dataDOI{}
\def \editorNAME{}
\def \editorORCID{}
\def \reviewerINAME{}
\def \reviewerIORCID{}
\def \reviewerIINAME{}
\def \reviewerIIORCID{}
\def \dateRECEIVED{01 November 2018}
\def \dateACCEPTED{}
\def \datePUBLISHED{}
\def \articleTITLE{[Re] The principal components of natural images}
\def \articleTYPE{Replication}
\def \articleDOMAIN{Computational Neuroscience}
\def \articleBIBLIOGRAPHY{bibliography.bib}
\def \articleYEAR{2020}
\def \reviewURL{}
\def \articleABSTRACT{The visual cortex has long been the subject of study for computational neuroscientists. Various neural network techniques have been applied in attempts to recreate the receptive fields found experimentally in visual cortex cells. In 1991 Hancock et al used an algorithm designed by Sanger in 1989 to perform Principal Component Analysis on a series of natural greyscale images, which replicated some features of receptive fields found in the V1 of cats, although they noted that many of the complex principal components were not found. Moreover, they showed that Sanger's algorithm was able to find distinguishing features in text images as well as natural images, an important find in the study of neural networks.}
\def \replicationCITE{Hancock, P. et al, 1991 The Principal Components of Natural Images}
\def \replicationBIB{}
\def \replicationURL{http://citeseerx.ist.psu.edu/viewdoc/download;jsessionid=044DFF560FC2E801B579C2F23D268B44?doi=10.1.1.41.192&rep=rep1&type=pdf}
\def \replicationDOI{10.1088/0954-898X/3/1/008}
\def \contactNAME{Iain Davies}
\def \contactEMAIL{id318@cam.c.uk}
\def \articleKEYWORDS{PCA, principal component analysis, MATLAB, V1, natural images, visual cortex}
\def \journalNAME{ReScience C}
\def \journalVOLUME{4}
\def \journalISSUE{1}
\def \articleNUMBER{}
\def \articleDOI{}
\def \authorsFULL{Iain Davies and Stephen Eglen}
\def \authorsABBRV{I. Davies and S. Eglen}
\def \authorsSHORT{Davies and Eglen}
\title{\articleTITLE}
\date{}
\author[1,\orcid{0000-0002-5361-6285}]{Iain Davies}
\author[1,\orcid{0000-0001-8607-8025}]{Stephen Eglen}
\affil[1]{Department of Applied Mathematics and Theoretical Physics, Cambridge University, Cambridge, UK}
